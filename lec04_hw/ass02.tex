\documentclass{article}
\usepackage{bm}
\usepackage{amsmath}
\usepackage{graphicx}
\usepackage{mdwlist}
\usepackage[colorlinks=true]{hyperref}
\usepackage{geometry}
\geometry{margin=1in}
\geometry{headheight=2in}
\geometry{top=1.0in}
\usepackage{palatino}
\usepackage{ulem}
\usepackage[parfill]{parskip}
%\renewcommand{\rmdefault}{palatino}
\usepackage{fancyhdr}
\usepackage{listings}
\usepackage{color}
\usepackage{framed}

\definecolor{background}{RGB}{39, 40, 34}
\definecolor{string}{RGB}{230, 219, 116}
\definecolor{comment}{RGB}{117, 113, 94}
\definecolor{normal}{RGB}{248, 248, 242}
\definecolor{identifier}{RGB}{166, 226, 46}



\lstset{
  language=C,               			% choose the language of the code
  alsolanguage=Python,            			% choose the language of the code
  alsolanguage=Java,            			% choose the language of the code
  numbers=none,                   		% where to put the line-numbers
  stepnumber=1,                   		% the step between two line-numbers.        
  numbersep=5pt,                  		% how far the line-numbers are from the code
  extendedchars=true,
  numberstyle=\tiny\color{black}\ttfamily,
  backgroundcolor=\color{background},  		% choose the background color. You must add \usepackage{color}
  showspaces=false,               		% show spaces adding particular underscores
  showstringspaces=false,         		% underline spaces within strings
  showtabs=false,                 		% show tabs within strings adding particular underscores
  frame=single,
  framerule=0pt,
  tabsize=4,                      		% sets default tabsize to 2 spaces
  captionpos=n,                   		% sets the caption-position to bottom
  breaklines=true,                		% sets automatic line breaking
  breakatwhitespace=true,         		% sets if automatic breaks should only happen at whitespace
  title=\lstname,                 		% show the filename of files included with \lstinputlisting;
  basicstyle=\color{normal}\tiny\ttfamily,					% sets font style for the code
  keywordstyle=\color{magenta}\tiny\ttfamily,	% sets color for keywords
  stringstyle=\color{string}\tiny\ttfamily,		% sets color for strings
  commentstyle=\color{comment}\tiny\ttfamily,	% sets color for comments
  emph={True, False, format_string, eff_ana_bf, permute, eff_ana_btr, KeyError,
  ValueError, ZeroDivisionError},
  emphstyle=\color{identifier}\tiny\ttfamily,
  morekeywords={with, as}
}

\lstset{literate=%
   *{0}{{{\color{cyan}0}}}1
    {1}{{{\color{cyan}1}}}1
    {2}{{{\color{cyan}2}}}1
    {3}{{{\color{cyan}3}}}1
    {4}{{{\color{cyan}4}}}1
    {5}{{{\color{cyan}5}}}1
    {6}{{{\color{cyan}6}}}1
    {7}{{{\color{cyan}7}}}1
    {8}{{{\color{cyan}8}}}1
    {9}{{{\color{cyan}9}}}1
}


%\pagestyle{fancy}

\newcommand{\infint}{\int_{-\infty}^{\infty}}
\newcommand{\Ab}{\bm{A}}
\rhead{}
\lhead{}
\chead{%
  {\vbox{%
      \vspace{1mm}
      \large
      Python practice 4\hfill \\
    }
  }
}

\usepackage{paralist}

\usepackage{todonotes}
\setlength{\marginparwidth}{2.15cm}

\usepackage{tikz}
\usetikzlibrary{positioning,shapes,backgrounds}

\newenvironment{enum}{
\begin{enumerate}
  \setlength{\itemsep}{1pt}
  \setlength{\parskip}{0pt}
  \setlength{\parsep}{0pt}
}{\end{enumerate}}

\begin{document}
\thispagestyle{fancy}
\setcounter{section}{-1}

%% Q4
\section{Chronicles of the Python: The File, the With and the Try}
The following code will create a file named $new.txt$, with a single line in
it. 
\begin{lstlisting}
  f = open("new.txt", 'w')
  f.write("Hello, World!")
  f.close()
\end{lstlisting}
The first argument of $open()$ is the filename, and the second argument of
$open()$ is open mode, with the following options(not all of them are listed).\\
\begin{center}
  \begin{tabular}{*2c}
    \hline
    'r' & Only read the file (When no options are given, this is the default mode)\\ \hline
    'w' & Newly write on the file (File with the same name will be
    deleted)\\ \hline
    'a' & Append at the end of the file\\ \hline
    'r+' & Read \& Write Mode (Error if file does not exist)\\ \hline
    'w+' & Read \& Write Mode (Create a new file if file does not exist)\\ \hline
    'rb', 'wb', \ldots & Open binary(non-text) files \\ \hline
  \end{tabular}\\
\end{center}
  After we are done with the file, we have to explicitly close it with
  $f.close()$, so that other processes can use the file.\\

  Write a program that would print the contents of a file named $log.txt$.
  After printing, it would add a new line containing the current time to the
  same file.
  \begin{enumerate}
    \item  Use the $with$ statement so that $f.close()$ can be automatically done. 
    \item Add appropriate error handling, with $try-except$, if necessary. Depends on which mode you
      choose to operate.
  \end{enumerate}
    \textit{hint}: You can get the current time by
    \begin{lstlisting}
  import time
  time.strftime('%H : %M')
    \end{lstlisting}
    Try out other options (\%Y, \%m, \%d \ldots) as well!
\section{Heart Signal}
It is not often the case that someone you love, loves you back. It was easily
observed in the heartwarming(or breaking?) show Heart Signal. Now, as the \#1 fan
of the show, it is our job to see if the participants can find the
perfect match. Perfect match means that the participants like each other.\\
\noindent\rule{\textwidth}{0.9pt}
\textit{INPUT}: The first line is integer $N$. The next $N$ lines contain the
name of all the participants, followed by the name of the person he/she likes,
seperated by a space. The last line is the name of a participant.

\textit{OUTPUT}: If the participant is also liked by the someone he/she likes,
print \textbf{LOVE}. Print \textbf{OUCH} elsewise. Print \textbf{WHO?} if the
name did not appear in the participant list.

\textit{EXAMPLE}:
\begin{lstlisting}
5
hulk blackwidow
hawkeye blackwidow
blackwidow hawkeye
ironman pepper
pepper ironman
hulk

OUCH
\end{lstlisting}

\begin{lstlisting}
5
hulk blackwidow
hawkeye blackwidow
blackwidow hawkeye
ironman pepper
pepper ironman
me

WHO?
\end{lstlisting}

%% Q2
\section{Statistics Showdown}
Given $N$($N \% 2 == 1$) numbers, compute the following representative values.
\begin{enumerate}
  \item Arithmetic Mean
  \item Geometric Mean
  \item Median
  \item Mode (The most common value)
  \item Range ($max(n_i) - min(n_i)$)
\end{enumerate}

\noindent\rule{\textwidth}{0.9pt}
\textit{INPUT}: The first line is integer $N$.
The \textit{i}-th line of next $N$ lines contains a single integer $n_{i}$.

\textit{OUTPUT}: Print 5 lines, each consisting of the Arithmetic Mean, Geometric
Mean, Median, Mode, and Range of $n_{i}$. If there are more than 1 mode, print
any of those. For Arithmetic and Geometric mean,
round to one decimal place (\textit{e.g.} $1.75 = 1.8, 3.74 = 3.7$).

\textit{CONDITION}: $ 1 \le N \le 200$ ($N$ is odd number),  $ 0 \le n_{i} \le 1000$

\textit{EXAMPLE}:
\begin{lstlisting}
5
1
4
2
3
5
\end{lstlisting}

\begin{lstlisting}
3.0
2.6
3
1
4
\end{lstlisting}

\textit{HINT}: Googling "root in python (other than square root)" would help
calculating the Geometric Mean.


%% Q1
\section{DIE with a T}
I am on a diet. I can only eat upto 2,000 kcal (2,000,000 cal) a day.
I eat 4 meals a day(\textit{b}reakfast, \textit{l}unch, \textit{s}nack, and
\textit{d}inner) and life would be easy if
every meal was exactly 500,000cal. But it isn't, so help me out.
From 4 different groups of menus(each group representing the 4 meals I eat)
each consisting of $N$ foods, compute whether it is possible to select a menu
from meal each group
so that the total calorie intake is \textit{exactly} 2,000,000 cal.\\
\noindent\rule{\textwidth}{0.9pt}
\textit{INPUT}: The first line is integer $N$.
The \textit{i}-th line of next $N$ lines contains the four calorie value of a
menu in each meal group. The first columns are breakfast menus, the
second lunch, and so on.

\textit{OUTPUT}: Print \textbf{SUCCESS} if we can choose exactly 4 food so that
the total of calorie equals 2,000,000.\\($b_{i}+l_{j}+s_{k}+d_{l} = 2,000,000$)
Print \textbf{FAIL} elsewise.

\textit{CONDITION}: $ 4 \le N \le 10,000$, $ 0 \le b,l,s,d \le 1,000,000$

\textit{EXAMPLE}:
\begin{lstlisting}
2
100000 600000 400000 600000
200000 500000 600000 800000
\end{lstlisting}
\begin{lstlisting}
SUCCESS
\end{lstlisting}
We can choose ($200000$, $600000$, $600000$, $600000$) to eat $2,000,000$cal.
\begin{lstlisting}
1
100000 600000 400000 600000
\end{lstlisting}
\begin{lstlisting}
FAIL
\end{lstlisting}

\textit{hint}: The obvious approach would be trying every combination
of $4$ foods from each meal group. However, this approach is $O(N^4)$,
taking time proportional to $10000000000000000$ in the worst case.
Therefore, instead of trying every combination, we need a better approach.
One possible approach would be computing \textit{partial} calorie $k$ in
$O(N^2)$, and checking whether $(2000000-k)$ is possible in $O(1)$.
$Constant$ time membership check can be done by a certain data structure, which
has the added benefit of removing duplicates, which will reduce computation
time somewhat.\\


\section{Optional: Parallelization}
Download $mat\_mul\_skeleton.py$ to begin.\\
We will use parallelization to speed up matrix multiplication, which is a very
common operation in modern computer programming. Of course, we will use the
simplest form of multiplication \& parallelization for convienience.

The key of parallelization is splitting the task into smaller, independent
tasks. Matrix multiplication can be split by multiplying the lefthand matrix with
each column of the righthand matrix:

\begin{align*}
  \begin{bmatrix}
    1 & 2\\
    3 & 4\\
  \end{bmatrix}
  &*
  \begin{bmatrix}
    5 & 6 & 7\\
    8 & 9 & 10\\
  \end{bmatrix}\\
  &\Downarrow\\
  \begin{bmatrix}
    1 & 2\\
    3 & 4\\
  \end{bmatrix}
  *
  \begin{bmatrix}
    5\\
    8\\
  \end{bmatrix}
  \&
  \begin{bmatrix}
    1 & 2\\
    3 & 4\\
  \end{bmatrix}
  &*
  \begin{bmatrix}
    6\\
    9\\
  \end{bmatrix}
  \&
  \begin{bmatrix}
    1 & 2\\
    3 & 4\\
  \end{bmatrix}
  *
  \begin{bmatrix}
    7\\
    10\\
  \end{bmatrix}
\end{align*}

We will use \textit{Pool().map(f,\ \ ll)} to parallelize the multiplication.
$f$ is a function that multiplies the lefthand matrix to a single
vector, and $ll$ is a $list$ of $list$ where each item($list$) is a column of
righthand matrix.

Other smarter representation of matrices like $numpy.array()$ provide column-wise
access. However, since we use $list$ of $list$ to represent matrices
column-wise access is frustrating. Therefore, we will transpose the matrix
so that each column now becomes the row. $transposed(matrix)[0]$ would return the
first column of the original matrix. We then apply f to each row of the
transposed matrix, then transpose it back.

I have implemented the $transpose$ and single-core matrix multiplication.
All we have to do is fill in the $\#TODO$, and try experiments, varying the size
of matrices. Which one is faster when matrix sizes are small? when large?
\end{document}
