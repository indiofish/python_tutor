\documentclass{beamer}
\usepackage{inconsolata}
\usepackage{color}
\usepackage{listings}
\usepackage{cooltooltips}
\usepackage[normalem]{ulem}
\setbeamertemplate{navigation symbols}{}%remove navigation symbols
\usepackage{listings}
\usepackage{color}
\usepackage{framed}

\definecolor{background}{RGB}{39, 40, 34}
\definecolor{string}{RGB}{230, 219, 116}
\definecolor{comment}{RGB}{117, 113, 94}
\definecolor{normal}{RGB}{248, 248, 242}
\definecolor{identifier}{RGB}{166, 226, 46}



\lstset{
  language=C,               			% choose the language of the code
  alsolanguage=Python,            			% choose the language of the code
  alsolanguage=Java,            			% choose the language of the code
  numbers=none,                   		% where to put the line-numbers
  stepnumber=1,                   		% the step between two line-numbers.        
  numbersep=5pt,                  		% how far the line-numbers are from the code
  extendedchars=true,
  numberstyle=\tiny\color{black}\ttfamily,
  backgroundcolor=\color{background},  		% choose the background color. You must add \usepackage{color}
  showspaces=false,               		% show spaces adding particular underscores
  showstringspaces=false,         		% underline spaces within strings
  showtabs=false,                 		% show tabs within strings adding particular underscores
  frame=single,
  framerule=0pt,
  tabsize=4,                      		% sets default tabsize to 2 spaces
  captionpos=n,                   		% sets the caption-position to bottom
  breaklines=true,                		% sets automatic line breaking
  breakatwhitespace=true,         		% sets if automatic breaks should only happen at whitespace
  title=\lstname,                 		% show the filename of files included with \lstinputlisting;
  basicstyle=\color{normal}\tiny\ttfamily,					% sets font style for the code
  keywordstyle=\color{magenta}\tiny\ttfamily,	% sets color for keywords
  stringstyle=\color{string}\tiny\ttfamily,		% sets color for strings
  commentstyle=\color{comment}\tiny\ttfamily,	% sets color for comments
  emph={True, False, format_string, eff_ana_bf, permute, eff_ana_btr, KeyError,
  ValueError, ZeroDivisionError},
  emphstyle=\color{identifier}\tiny\ttfamily,
  morekeywords={with, as}
}

\lstset{literate=%
   *{0}{{{\color{cyan}0}}}1
    {1}{{{\color{cyan}1}}}1
    {2}{{{\color{cyan}2}}}1
    {3}{{{\color{cyan}3}}}1
    {4}{{{\color{cyan}4}}}1
    {5}{{{\color{cyan}5}}}1
    {6}{{{\color{cyan}6}}}1
    {7}{{{\color{cyan}7}}}1
    {8}{{{\color{cyan}8}}}1
    {9}{{{\color{cyan}9}}}1
}



\hypersetup{
  colorlinks=true,
  urlcolor=pink,
}

\title{Python 101}
\subtitle{Lec02 \\ Ifs and Loops}
\author{thoum}

\begin{document}
\frame{\titlepage}

\begin{frame}[fragile]
\frametitle{Possible Projects}
  \begin{enumerate}
    \item Beating sugang.snu.ac.kr's CAPTCHA\\
      - Digit recognition with kNN
    \item Music Recognition with kNN
    \item JoonggoNara Notifier
  \end{enumerate}
\end{frame}

\begin{frame}[fragile]
\frametitle{If}
\begin{lstlisting}
if some_boolean:
    pass #do something
\end{lstlisting}
\end{frame}

\begin{frame}
\frametitle{If examples}
  \begin{lstinputlisting}
    {./grades.py}
  \end{lstinputlisting}
  \href{https://stackoverflow.com/questions/48375753/why-are-chained-operator-expressions-slower-than-their-expanded-equivalent}{*why?}
\end{frame}

\begin{frame}
\frametitle{More examples}
  \begin{lstinputlisting}
    {./if_example.py}
  \end{lstinputlisting}
\end{frame}

\begin{frame}
\frametitle{and more examples}
  \begin{lstinputlisting}
    {./if_example2.py}
  \end{lstinputlisting}
\end{frame}

\begin{frame}{For Loops}
  \sout{Repeat N times}

  Iterate members of a sequence.
\end{frame}

\begin{frame}{For Loops, C style}
  \begin{lstinputlisting}
    {./for.c}
  \end{lstinputlisting}
\end{frame}

\begin{frame}[fragile]
  \begin{lstlisting}
for value in iterable:
    pass
  \end{lstlisting}
\end{frame}

\begin{frame}{For Loops}
  \begin{lstinputlisting}[firstline=1, lastline=14]
    {./for_default.py}
  \end{lstinputlisting}
\end{frame}

\begin{frame}{For Loops Cont'}
  \begin{lstinputlisting}[firstline=1, lastline=14]
    {./for_default.py}
  \end{lstinputlisting}
\end{frame}

\begin{frame}{For Loops Nested}
  \begin{lstinputlisting}
    {./for_nest.py}
  \end{lstinputlisting}
\end{frame}

\begin{frame}{While Loops}
  Repeat \textit{while} condition(boolean) is satisfied.
\end{frame}

\begin{frame}[fragile]
  \begin{lstlisting}
while condition:
    pass
  \end{lstlisting}
\end{frame}

\begin{frame}[fragile]
  \begin{lstinputlisting}
    {./while.py}
  \end{lstinputlisting}
\end{frame}

\begin{frame}{Practice}
  Output: print 99dan from 1 to N.
  Hint: \textcolor{white}{Nest for loops}
\end{frame}


\begin{frame}[fragile]{Practice with stars!}
  Input: N

  Print:
  \begin{lstlisting}
*
**
***

  *
 **
***

  *
 ***
*****
  \end{lstlisting}
  Hint: \textcolor{white}{'*' * k}
\end{frame}


\begin{frame}[fragile]
  Using for loops, create list of even numbers in [0, 10]
  \begin{lstlisting}
lst = []
for n in range(0, 11, 2):
    lst.append(n)

# or

for n in range(0, 11):
    if n % 2 == 0:
        lst.append(n)
  \end{lstlisting}
\end{frame}

\begin{frame}[fragile]{Comprehension}
  The easy way (and faster!)
  \begin{lstlisting}
  lst = [n for n in range(11) if n % 2==0]
  \end{lstlisting}
\end{frame}

\begin{frame}[fragile]{Comprehension}
  new\_list = [expression $for$ element $in$ iterable ($if$ condition)]\\
  \begin{lstlisting}
for element in iterable:
    if condition:
      new_list.append(expression)
  \end{lstlisting}
\end{frame}

\begin{frame}[fragile]{Comprehension}
  \begin{lstinputlisting}
    {./comp.py}
  \end{lstinputlisting}
\end{frame}

\begin{frame}[fragile]{Generators}
  Lazy version of comprehension [] $\rightarrow$ ()
\end{frame}

\begin{frame}{Practice}
  Nth Fibonacci
\end{frame}

\begin{frame}{Practice}
  Nth Fibonacci
\end{frame}






\end{document}
