\documentclass{article}
\usepackage{bm}
\usepackage{amsmath}
\usepackage{graphicx}
\usepackage{mdwlist}
\usepackage[colorlinks=true]{hyperref}
\usepackage{geometry}
\geometry{margin=1in}
\geometry{headheight=2in}
\geometry{top=2in}
\usepackage{palatino}
%\renewcommand{\rmdefault}{palatino}
\usepackage{fancyhdr}
\usepackage{listings}
\usepackage{color}
\usepackage{framed}

\definecolor{background}{RGB}{39, 40, 34}
\definecolor{string}{RGB}{230, 219, 116}
\definecolor{comment}{RGB}{117, 113, 94}
\definecolor{normal}{RGB}{248, 248, 242}
\definecolor{identifier}{RGB}{166, 226, 46}



\lstset{
  language=C,               			% choose the language of the code
  alsolanguage=Python,            			% choose the language of the code
  alsolanguage=Java,            			% choose the language of the code
  numbers=none,                   		% where to put the line-numbers
  stepnumber=1,                   		% the step between two line-numbers.        
  numbersep=5pt,                  		% how far the line-numbers are from the code
  extendedchars=true,
  numberstyle=\tiny\color{black}\ttfamily,
  backgroundcolor=\color{background},  		% choose the background color. You must add \usepackage{color}
  showspaces=false,               		% show spaces adding particular underscores
  showstringspaces=false,         		% underline spaces within strings
  showtabs=false,                 		% show tabs within strings adding particular underscores
  frame=single,
  framerule=0pt,
  tabsize=4,                      		% sets default tabsize to 2 spaces
  captionpos=n,                   		% sets the caption-position to bottom
  breaklines=true,                		% sets automatic line breaking
  breakatwhitespace=true,         		% sets if automatic breaks should only happen at whitespace
  title=\lstname,                 		% show the filename of files included with \lstinputlisting;
  basicstyle=\color{normal}\tiny\ttfamily,					% sets font style for the code
  keywordstyle=\color{magenta}\tiny\ttfamily,	% sets color for keywords
  stringstyle=\color{string}\tiny\ttfamily,		% sets color for strings
  commentstyle=\color{comment}\tiny\ttfamily,	% sets color for comments
  emph={True, False, format_string, eff_ana_bf, permute, eff_ana_btr, KeyError,
  ValueError, ZeroDivisionError},
  emphstyle=\color{identifier}\tiny\ttfamily,
  morekeywords={with, as}
}

\lstset{literate=%
   *{0}{{{\color{cyan}0}}}1
    {1}{{{\color{cyan}1}}}1
    {2}{{{\color{cyan}2}}}1
    {3}{{{\color{cyan}3}}}1
    {4}{{{\color{cyan}4}}}1
    {5}{{{\color{cyan}5}}}1
    {6}{{{\color{cyan}6}}}1
    {7}{{{\color{cyan}7}}}1
    {8}{{{\color{cyan}8}}}1
    {9}{{{\color{cyan}9}}}1
}


%\pagestyle{fancy}

\newcommand{\infint}{\int_{-\infty}^{\infty}}
\newcommand{\Ab}{\bm{A}}
\rhead{}
\lhead{}
\chead{%
  {\vbox{%
      \vspace{2mm}
      \large
      Python practice 1\hfill
\\
    }
  }
}

\usepackage{paralist}

\usepackage{todonotes}
\setlength{\marginparwidth}{2.15cm}

\usepackage{tikz}
\usetikzlibrary{positioning,shapes,backgrounds}

\newenvironment{enum}{
\begin{enumerate}
  \setlength{\itemsep}{1pt}
  \setlength{\parskip}{0pt}
  \setlength{\parsep}{0pt}
}{\end{enumerate}}

\begin{document}
\pagestyle{fancy}
\setcounter{section}{-1}

%% Q1
\section{Getting Inputs}
There are many kind of inputs, but we will focus on numbers for now.
Python have to deal with these two types of inputs differently.
\begin{enumerate}
  \item One Line
    \begin{lstlisting}
1 2 3 4 5...
    \end{lstlisting}
  There is \textit{much} cleaner way to do this, which we will learn next week.
  \begin{lstlisting}
lst = input().split() # note that after split(), type is lst.
nums = [] * len(lst)
for i in range(len(nums)):
    nums[i] = int(lst(i))
  \end{lstlisting}
  \item Seperate Lines
  \begin{lstlisting}
1
2
3
4
5
...
  \end{lstlisting}
  We can do this by using For Loops. Usually, the number of inputs is also given.
  The following code can store numbers given in this manner.
  \begin{lstlisting}
k = int(input()) # number of inputs
lst = [0] * k

for i in range(k):
    lst[i] = int(input())
  \end{lstlisting}
\end{enumerate}

\section{Lists}
Generate a list of numbers [1, 100] with range.
Print the value of $1 * 2 * ...99 * 100$.

\section{99 dan}
Print out the following:
\begin{lstlisting}
1 * 1 = 1
1 * 2 = 2
   ...
9 * 9 = 81
\end{lstlisting}
We \textit{can} construct the string by concatanation like this:
\begin{lstlisting}
str(8) + ' * ' + str(9) + ' = ' + str(72)
\end{lstlisting}
but there is a better way.
\href{https://www.google.com/search?q=python+string+format}{Google It!}

\section{Lotto}
Create and print a list of length \textit{6}, where each element is a random
integer in range $[1, 45]$.

\begin{lstlisting}
lst = [0] * 6 # create a list of length 6, with all the elements set to 0.

# Do Something

print(*lst) # remember unpacking?
\end{lstlisting}
To learn how to generate random numbers
\href{https://www.google.com/search?q=python+random+number}{Google It!}
Among all the search result google provides, use \href{https://stackoverflow.com}{stackoverflow}.


\section{Buy Lotto}
Create a list of length \textit{6}, where each element is a random integer in
range $[1, 45]$.  Now, get input from user(you) until the user has correctly typed the 6
numbers.\\
Note that order of numbers do not matter and CONFIRM that your program finishes correctly.


\section{369}
Print numbers in [1, 50]. But for numbers that are multiples of \textit{3}  ($e.g.$
3,6,9...) and numbers that contain(\textit{membership??}) \textit{3}  ($e.g.$ 13, 23...), print clap.
\begin{lstlisting}
...
11
clap
clap
14
clap
16
...
\end{lstlisting}

\section{2nd}
Create a list of 100 random numbers.\\
output: the 2nd largest element from the numbers.\\
Sure, we can do the following. But solve it without sorting the list.
\begin{lstlisting}
print(sorted(lst[-2]))
\end{lstlisting}

\pagebreak
\section{Challange: Fibonacci}
The Fibonacci numbers are defined as following:
$$F_0 = 0, F_1 = 1$$
$$F_n = F_{n-1}+F_{n-2}, n > 1$$
Input: N, Output: $N^th$ Fibonacci.\\
Check that this program also prints $F_0$ and $F_1$ well, for input 0 and 1.\\
Hint: We probably should store $F_0$ and $F_1$ somewhere, and work with them.

\section{Challange: Matching parentheses Strings}
Mismatched ()s ($e.g.$ "(()", ")(()" ) are painful when we are coding. Write a
program that prints 'yes' when the ()s match, and 'no' when they do not match.

\begin{lstlisting}
())()
no
(())()
yes
\end{lstlisting}
Googling this would probably lead you to the stack data structure. It is pretty
simple to implement a stack using python lists. Try to figure it out.
But, we do not need a stack to solve this one.

\end{document}
