\documentclass{beamer}
\usepackage{inconsolata}
\usepackage{caption}
\usepackage{color}
\usepackage{listings}
\usepackage{subfig}
\usepackage{cooltooltips}
\usepackage{hyperref}
\usepackage{perpage}
\usepackage[normalem]{ulem}
\setbeamertemplate{navigation symbols}{}%remove navigation symbols
\usepackage{listings}
\usepackage{color}
\usepackage{framed}

\definecolor{background}{RGB}{39, 40, 34}
\definecolor{string}{RGB}{230, 219, 116}
\definecolor{comment}{RGB}{117, 113, 94}
\definecolor{normal}{RGB}{248, 248, 242}
\definecolor{identifier}{RGB}{166, 226, 46}



\lstset{
  language=C,               			% choose the language of the code
  alsolanguage=Python,            			% choose the language of the code
  alsolanguage=Java,            			% choose the language of the code
  numbers=none,                   		% where to put the line-numbers
  stepnumber=1,                   		% the step between two line-numbers.        
  numbersep=5pt,                  		% how far the line-numbers are from the code
  extendedchars=true,
  numberstyle=\tiny\color{black}\ttfamily,
  backgroundcolor=\color{background},  		% choose the background color. You must add \usepackage{color}
  showspaces=false,               		% show spaces adding particular underscores
  showstringspaces=false,         		% underline spaces within strings
  showtabs=false,                 		% show tabs within strings adding particular underscores
  frame=single,
  framerule=0pt,
  tabsize=4,                      		% sets default tabsize to 2 spaces
  captionpos=n,                   		% sets the caption-position to bottom
  breaklines=true,                		% sets automatic line breaking
  breakatwhitespace=true,         		% sets if automatic breaks should only happen at whitespace
  title=\lstname,                 		% show the filename of files included with \lstinputlisting;
  basicstyle=\color{normal}\tiny\ttfamily,					% sets font style for the code
  keywordstyle=\color{magenta}\tiny\ttfamily,	% sets color for keywords
  stringstyle=\color{string}\tiny\ttfamily,		% sets color for strings
  commentstyle=\color{comment}\tiny\ttfamily,	% sets color for comments
  emph={True, False, format_string, eff_ana_bf, permute, eff_ana_btr, KeyError,
  ValueError, ZeroDivisionError},
  emphstyle=\color{identifier}\tiny\ttfamily,
  morekeywords={with, as}
}

\lstset{literate=%
   *{0}{{{\color{cyan}0}}}1
    {1}{{{\color{cyan}1}}}1
    {2}{{{\color{cyan}2}}}1
    {3}{{{\color{cyan}3}}}1
    {4}{{{\color{cyan}4}}}1
    {5}{{{\color{cyan}5}}}1
    {6}{{{\color{cyan}6}}}1
    {7}{{{\color{cyan}7}}}1
    {8}{{{\color{cyan}8}}}1
    {9}{{{\color{cyan}9}}}1
}



\newenvironment{enum}{
\begin{enumerate}
  \setlength{\itemsep}{1pt}
  \setlength{\parskip}{0pt}
  \setlength{\parsep}{0pt}
}{\end{enumerate}}

\hypersetup{
  colorlinks=true,
  urlcolor=pink,
}

\MakePerPage{footnote}

\title{Python 101}
\subtitle{Lec07 \\ Classes Continued}
\author{thoum}

\begin{document}
\frame{\titlepage}

\begin{frame}
\frametitle{Inheritance}
One usage of classes is Inheritance.
\end{frame}

\begin{frame}
\frametitle{Inheritance}
The child inherits every thing about its parent, and $+\alpha$.
\end{frame}

\begin{frame}[fragile]
\frametitle{Inheritance}
\begin{lstinputlisting}
  {./mylist.py}
\end{lstinputlisting}
\end{frame}

\begin{frame}{Explained}
  $MyList(list)$ means that this class will inherit from $list$.
  \begin{lstinputlisting}[firstline=1, lastline=1]
  {./mylist.py}
\end{lstinputlisting}
\end{frame}

\begin{frame}{Explained}
  The $super()$ returns the parent class. We use super() to access the parent
  classes data methods etc. Here, we initiate the parent first so that
  parameters are automatically filled in.
  \begin{lstinputlisting}[firstline=1, lastline=4]
  {./mylist.py}
\end{lstinputlisting}
\end{frame}

\begin{frame}{Explained}
  $min(lst), max(lst)$ takes time proprotional to N.\\
  Here, we keep track of min and max so that it can be known regardless of
  size.\\
  (Of course, there is no free lunch, there is extra cost of comparing at append.)
  \begin{lstinputlisting}[firstline=1, lastline=6]
  {./mylist.py}
  \end{lstinputlisting}
\end{frame}

\begin{frame}{Explained}
  Here, we override the parent's \_\_len\_\_, which determines the value
  returned when we do $len(lst)$.
  \begin{lstinputlisting}[firstline=8, lastline=10]
    {./mylist.py}
  \end{lstinputlisting}
\end{frame}

\begin{frame}{Explained}
  We override the append() of list, so that we keep track of \_min and \_max.\\
  After updating \_min and \_max, we insert to the list via super() call.
  \begin{lstinputlisting}[firstline=12, lastline=17]
    {./mylist.py}
  \end{lstinputlisting}
\end{frame}

\begin{frame}{Question}
  Are we done implementing MyList so that it correctlys keep track of min and max?
\end{frame}

\begin{frame}{Question}
  \begin{center}
    NO
  \end{center}
\end{frame}

\begin{frame}{The Catch}
  Everything that can be done to a $list$ can be done to a $MyList$.\\
  $pop(),\ del,\ insert(),\ mylst[3]=4,\ mylst[3:4]$... you name it.\\
  \mbox{}\\
  This means that to correctly keep track of min \& max, we have to override
  \textit{every single} method of a list that is capable of changing its
  contents.\\
  \mbox{}\\
  Can you remember all of them?
\end{frame}

\begin{frame}{Composition over Inheritance\footnote{Look this up in Google}}
    So, it is often wise to compose your class with a list, rather than inheriting
    it.
\end{frame}

\begin{frame}{MyList Composition Ver.}
  We can control the methods we provide for interaction with the internal list.
  \begin{lstinputlisting}[firstline=1, lastline=23]
    {./composition.py}
  \end{lstinputlisting}
\end{frame}

\begin{frame}{Inheritance?}
  We might use inheritance when the child has to provide every method its parents
  provide + $\alpha$. \\($i.e.$ The Gun class that inherits the Weapon class in an RPG
  game?). \\But Design Patterns are a complicated subject by itself, so we won't
  deal it in detail.
\end{frame}

\begin{frame}{Questions?}
\end{frame}

\begin{frame}{Questions on homeworks?}
\end{frame}

\begin{frame}{Assignment}
  Finish implementing myList using the composition version.
  It should support $len(mylst), mylst[3]=2, and mylst.append()$, updating \_min
  and \_max accordingly.
\end{frame}

\begin{frame}{For Next Week}
  \begin{center}
    Twitter Crawler
  \end{center}
\end{frame}

\begin{frame}{Demonstration}
  (I will not post the code due to security reasons.)
\end{frame}

\begin{frame}{For Next Week}
  If we have time, we do the prerequisites together.
  Else, it's homework.
\end{frame}
\end{document}
