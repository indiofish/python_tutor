\documentclass{article}
\usepackage{bm}
\usepackage{amsmath}
\usepackage{graphicx}
\usepackage{mdwlist}
\usepackage[colorlinks=true]{hyperref}
\usepackage{geometry}
\geometry{margin=1in}
\geometry{headheight=2in}
\geometry{top=2in}
\usepackage{palatino}
\usepackage{ulem}
\usepackage[parfill]{parskip}
%\renewcommand{\rmdefault}{palatino}
\usepackage{fancyhdr}
\usepackage{listings}
\usepackage{color}
\usepackage{framed}

\definecolor{background}{RGB}{39, 40, 34}
\definecolor{string}{RGB}{230, 219, 116}
\definecolor{comment}{RGB}{117, 113, 94}
\definecolor{normal}{RGB}{248, 248, 242}
\definecolor{identifier}{RGB}{166, 226, 46}



\lstset{
  language=C,               			% choose the language of the code
  alsolanguage=Python,            			% choose the language of the code
  alsolanguage=Java,            			% choose the language of the code
  numbers=none,                   		% where to put the line-numbers
  stepnumber=1,                   		% the step between two line-numbers.        
  numbersep=5pt,                  		% how far the line-numbers are from the code
  extendedchars=true,
  numberstyle=\tiny\color{black}\ttfamily,
  backgroundcolor=\color{background},  		% choose the background color. You must add \usepackage{color}
  showspaces=false,               		% show spaces adding particular underscores
  showstringspaces=false,         		% underline spaces within strings
  showtabs=false,                 		% show tabs within strings adding particular underscores
  frame=single,
  framerule=0pt,
  tabsize=4,                      		% sets default tabsize to 2 spaces
  captionpos=n,                   		% sets the caption-position to bottom
  breaklines=true,                		% sets automatic line breaking
  breakatwhitespace=true,         		% sets if automatic breaks should only happen at whitespace
  title=\lstname,                 		% show the filename of files included with \lstinputlisting;
  basicstyle=\color{normal}\tiny\ttfamily,					% sets font style for the code
  keywordstyle=\color{magenta}\tiny\ttfamily,	% sets color for keywords
  stringstyle=\color{string}\tiny\ttfamily,		% sets color for strings
  commentstyle=\color{comment}\tiny\ttfamily,	% sets color for comments
  emph={True, False, format_string, eff_ana_bf, permute, eff_ana_btr, KeyError,
  ValueError, ZeroDivisionError},
  emphstyle=\color{identifier}\tiny\ttfamily,
  morekeywords={with, as}
}

\lstset{literate=%
   *{0}{{{\color{cyan}0}}}1
    {1}{{{\color{cyan}1}}}1
    {2}{{{\color{cyan}2}}}1
    {3}{{{\color{cyan}3}}}1
    {4}{{{\color{cyan}4}}}1
    {5}{{{\color{cyan}5}}}1
    {6}{{{\color{cyan}6}}}1
    {7}{{{\color{cyan}7}}}1
    {8}{{{\color{cyan}8}}}1
    {9}{{{\color{cyan}9}}}1
}


%\pagestyle{fancy}

\newcommand{\infint}{\int_{-\infty}^{\infty}}
\newcommand{\Ab}{\bm{A}}
\rhead{}
\lhead{}
\chead{%
  {\vbox{%
      \vspace{2mm}
      \large
      Python practice 2\hfill
\\
    }
  }
}

\usepackage{paralist}

\usepackage{todonotes}
\setlength{\marginparwidth}{2.15cm}

\usepackage{tikz}
\usetikzlibrary{positioning,shapes,backgrounds}

\newenvironment{enum}{
\begin{enumerate}
  \setlength{\itemsep}{1pt}
  \setlength{\parskip}{0pt}
  \setlength{\parsep}{0pt}
}{\end{enumerate}}

\begin{document}
\pagestyle{fancy}
\setcounter{section}{-1}

%% Q1
\section{DIE with a T}

I am on a diet. That means I can only eat upto 2,000 kcal (2,000,000 cal) a day.
I eat 4 meals a day(\textit{b}reakfast, \textit{l}unch, \textit{s}nack, and
\textit{d}inner) and life would be easy if
every food was exactly 500,000cal. But it isn't. So help me out.
From 4 different groups of foods(each for the 4 meals I eat) each consisting of $N$
foods, compute whether it is possible to select a food from each group
so that the total calorie intake is \textit{exactly} 2,000,000 cal.\\
\noindent\rule{\textwidth}{0.5pt}
\textit{INPUT}: The first line is integer $N$.
The \textit{i}-th line of next $N$ lines contains the four calorie value of each
meal, $b_{i}, l_{i}, s_{i}, d_{i}$.

\textit{OUTPUT}: Print \textbf{SUCCESS} if we can choose exactly 4 food so that
the total of calorie equals 2,000,000.\\($b_{i}+l_{i}+s_{i}+d_{i} = 2,000,000$)
Print \textbf{FAIL} elsewise.

\textit{CONDITION}: $ 4 \le N \le 10,000$, $ 0 \le b,l,s,d \le 1,000,000$

\textit{HINT}: The obvious approach would be to try every combination
of $4$ foods from each meal. However, this approach takes time proportional to $N^4$,
which is $10000000000000000$ in the worst case.
Therefore, instead of trying every combination, we need a better approach.
One possible approach would be computing \textit{partial} calorie $k$ in time
proportional to $N^2$, and checking whether $2,000,000-k$ exists in $constant$ time. This can be done by a $set$. Using a $set$ also has the added benefit of
\rule{0.5cm}{0.15mm} \rule{0.7cm}{0.15mm}, reducing computation time somewhat.\\
\noindent\rule{\textwidth}{0.5pt}

%% Q2
\section{Statistics Showdown}
Given $N$($N \% 2 == 1$) numbers, compute the following representative values.
\begin{enumerate}
  \item Arithmetic Mean
  \item Geometric Mean
  \item Median
  \item Mode (The most common value)
  \item Range (max - min)
\end{enumerate}

\noindent\rule{\textwidth}{0.5pt}
\textit{INPUT}: The first line is integer $N$.
The \textit{i}-th line of next $N$ lines contains a single integer $n_{i}$.

\textit{OUTPUT}: Print 5 lines, each consisting of the Arithmetic Mean, Geometric
Mean, Median, Mode, and Range. For Arithmetic and Geometric mean, round to one
decimal place($1.75 = 1.8, 3.74 = 3.7$).

\textit{CONDITION}: $ 1 \le N \le 200$, $ 0 \le n_{i} \le 1000$

\textit{HINT}: Googling "root in python (other than square root)" would help
calculating the Geometric Mean.
\noindent\rule{\textwidth}{0.5pt}

%% Q3
\section{Prefix Calculator}
We usually write mathematical expressions this way:
$3+4*2$. This is called an infix notation, and although easy to understand, it has
some drawbacks.
%\begin{enumerate}
  %\item We have to know precedence rules of operators to correctly compute the
    %expression. Without knowing that '*' comes before '+', $3+4*2$ can be
    %miscalculated as $14$.
  %\item We have to use parentheses to express the intended order of operations.
    %$(3+4)*2$ if we want the result to be 14.
%\end{enumerate}
Therefore, there are other notations like prefix notations(operators precede
their operands), or postfix notations(operators follow their operands).
For example, using prefix notation, the expression above becomes $(+\ 3\ (*\ 4\
2))$.
To evaluate this expression, we start from the left, and when we see an
operator('+'), we expect two sub-expressions. We see a 3 and (* 4 2).
3 is 3. For (* 4 2), we repeat the step above, with an operator('*') and
two sub-expressions (4 and 2). Since there are no more sub-expressions to be
evaluated, we calculate $(*\ 4\ 2) = 8$, and $(+\ 3\ 8) = 11$.

TLDR; we are going to implement a mathematical expression
evaluator. For your convenience, I have already implemented infix to prefix function.

It's your job fill in $myeval(exp)$ to evaluate the prefix notation.


\noindent\rule{\textwidth}{0.5pt}
\textit{INPUT}: The first line is integer $N$.
The \textit{i}-th line of next $N$ lines contains a single integer $n_{i}$.

\textit{OUTPUT}: Print 5 lines, each consisting of the Arithmetic Mean, Geometric
Mean, Median, Mode, and Range. For Arithmetic and Geometric mean, round to one
decimal place($1.75 = 1.8, 3.74 = 3.7$).

\textit{CONDITION}: $ 1 \le N \le 200$, $ 0 \le n_{i} \le 1000$

\textit{HINT}: The return value of infix\_prefix is formed as following.
\begin{alignat*}{3}
  exp &::= \ \  &&(operand, exp, exp)\ \  &&tuple\\
      &\ \ \ | &&n &&integer
\end{alignat*}
So, when the type of the expression is integer, (in python: $isinstance(exp, int)$), there
is nothing to calculate and myeval just returns $n$. This becomes the basecase
of recursion. For the tuple case, assuming that $myeval(exp)$ correctly
calculates the results of the two subexpression would help.\\
\noindent\rule{\textwidth}{0.5pt}




\end{document}

























